\documentclass[12pt]{article}
\usepackage{graphicx}
\usepackage{pdfpages}
\usepackage{hyperref}
\usepackage[margin=1in]{geometry}
\usepackage[utf8]{inputenc}
\graphicspath{ {./Figures/} }
\hypersetup{
    colorlinks=true,
    linkcolor=blue,
    filecolor=magenta,      
    urlcolor=cyan,
}
\urlstyle{same}
\usepackage[font={small,it}]{caption}
\usepackage{fancyvrb}
\title{Analysis, Visualization, \& Simulation of South Bend Government Call Center}
\author{John D. Bulger\\Valparaiso University}
\date{March 10, 2019}

	\begin{document}
\maketitle

\section{Introduction}
The city of South Bend, located in northern Indiana, established a citizen-accessible call center in February 2013.  It addresses almost every aspect of city-citizen interaction, including waste pick-up and removal, water billing and disconnections, and code enforcement.  By serving as a central hub for communication, the call center is able to consolidate a substantial amount of data regarding citizens as consumers.  This data is available on South Bend's open data portal at \textit{https://data-southbend.opendata.arcgis.com}.

\subsection{Goals \& Desired Results}

% Clean up and put stuff here

\section{Prior Work}
	
Efficiency and success of a call center can be measured by different metrics.  Baraka identifies success principles such as customer satisfaction, service quality, and usage amount \cite{baraka}. Baraka specifically suggests call duration as a quantifiable, key metric \cite{baraka}. Other work on call centers have focused on first call resolution (“FCR”), customer satisfaction, and operator tenure as a key metrics, albeit with differing analytical goals.  Chen et al. use FCR in order to analyze call center efficiency utilizing rough set theory \cite{chen}. Within in the scope of this paper’s analysis, and given what information can be gleaned from the raw data, call duration will be a focus for discovering trends and insights.
\par
Another foundation for understanding a call center’s flow is an examination of call arrival distributions. While this analysis does not theoretically explore call arrival distribution, a brief discussion is still relevant to this work. Brown et al. provide an excellent overview of the Poisson distribution for modeling customer arrivals (in this case, incoming calls to the call center) \cite{brown}.  Brown et al. also discuss the necessary accompanying assumptions for this arrival distribution to hold true. Of these, the assumption that customers and operators are statistically identical and that they all act independently are most important in this paper’s analysis. Zhang, Hong, and Zhang also describe the call arrivals as a Poisson distribution, but discuss models which may provide more accurate alternatives \cite{zhang}. They maintain some of Brown et al.’s assumptions. Based on the results in these two papers, differences in time-dependent parameters, customer attitudes and preferences, and operator skill levels are treated as negligible in this analysis.
\par
Given the potential complexity of parameters, distributions, and desired metrics already described, call centers are a prime opportunity to utilize simulation techniques.  In fact, according to Bapat and Pruitte, simulation is the preferred method to analyze and determine the effectiveness of a call center \cite{bapat}.  Simulation can allow for evaluation of metrics beyond what a basic analysis encompasses.  For example, the scheduling of call agents should be optimized against call duration and abandonment, as documented by Saltzman and Mehrota \cite{saltzman}.  Such simulations can easily be created to illustrate feasibility of goal accomplishment for corporations while enabling efficient and accurate decision-making\cite{saltzman2}.


\section{Data Analysis}

The data to be analyzed was acquired in three distinct formats (files).

\paragraph{Daily Data} contains a daily summary of the call center data from the years 2013-2015.  It has 637 records indexed by date, and each record contains information such as the number of calls received, average wait time, average number of calls in queue, and the number of abandoned calls for that day.

\paragraph{Case Data} contains 488,601 records of individual call data from 2013 through 2015.  Calls were logged anonymously with data such as date, duration, topic, and department.  This data format went obsolete when the current call system was implemented.

\paragraph{Topic Data} contains many of the same attributes as the case data, but it reflects  the city's current use of knowledge base articles.  This systematizes the format for recording issues for each call, since each topic has a standardized name.  This ensures that topics and departments match across the list, allowing for efficient and accurate analysis.  This stands in contrast to the format of the older Case Data.

\par

After loading the data, it was inspected for missing data and reasonableness.  The daily data was missing no data points.  In the case data, less than 4\% of the observations contained missing data.  Upon further inspection, it was found that these rows were missing a significant proportion of their respective attributes and thus were dropped completely.  The topic data was found to be missing values in nearly 90\% of the entries in an attribute used for open-format comments.  Since a substantial amount was missing and it was irrelevant to the planned analysis, this attribute was dropped completely.  Preprocessing and transformation were conducted to transform the attributes into more useful Python data types, such as date-time and numeric objects.

	\subsection{Analytical Methods}
	



























































\section{Visualization}



\section{Mathematical Model}

	\subsection{Base Model}
The core model structure is based off the work of Mandelbaum in 2001\cite{mandelbaum}.  In his highly esteemed text, Mandelbaum lays out the basic schematic of a call center model.  His model illustrates the possible flow of a call, starting with its arrival until disposal, be that as a lost call or having completed successfully.  While simplistic by nature, the model covers all of the basic aspects of a call center in a sensible format that can easily by applied in Arena.  This model can be seen in Figure 2.  

	\begin{figure}[h]
	\includegraphics[scale=.45]{call_center_layout.png}
	\caption{Basic Operational Schematic of Call Center (Mandelbaum 2001)}
	\end{figure}

Mandelbaum's model identifies one avenue for arrivals (incoming calls) and three methods of disposal(lost calls due to busy signal, lost calls due to wait time, and completed calls).  It can be seen that the lost calls due to busy signals and the lost calls due to wait time both have retrial loops built in where some customers may immediately try to call into the system again and are treated as new arrivals for the sake of the simulation.  Completed calls also have the option to ``return", where they would immediately re-enter the system for whatever reason (perhaps an unresolved issue).  There is a single centralized queue before the calls are distributed among resources (operators).  At its core, this is a condensed high-level view of a call center model, specifically designed so that it could easily be expanded as needed for specific analyses.

\subsection{Analysis-Specific Model}
The model used for this analysis expands a great deal on the core model identified by Mandelbaum.  It adds three key features that the original model omits: voicemails, operator breaks, and an alternate call flow for supervisor assistance.  Additionally, calls are assigned key attributes that determine their specific flow and timing throughout the simulation.  These key attributes include topic, language, and priority.  The hypothetical self-service option was added on to this expanded model for the relevant analysis.

	\paragraph{Voicemails}
	
In addition to arriving calls, voicemails are ``created" as soon as each simulation ``day" begins, so that the call center has all day to return those calls.  These are considered a lower priority item for the call center operators.  A constant number of voicemails are received by the call center immediately at the beginning of each simulated day to represent the voicemails that built up the night before.

	\paragraph{Operator Breaks}
	
To try and obtain more accurate results, the operator resources were given ``breaks" to better simulate potential slow downs due to operator time off. Break scheduling is different for full-time and part-time employees, so both were taken into account for the simulation.

	\paragraph{Supervisor}
	
Some calls get routed to a supervisor after speaking with an agent.  This supervisor resource can then resolve the call, or pass it back to an agent.  This was added in an attempt to model actual call center behavior, where some callers may need assistance beyond a basic operator or may request supervisor assistance for a dispute. 

	\paragraph{Topic}

Calls are assigned a topic based on the distribution of topics in historical call center data.  In an effort to preserve the simplicity of the model while still utilizing past data, the top six departments (by total number of calls) are specified, with the remaining calls grouped into an ``other" category.  By assigning a topic, the simulation will be able to utilize the historical mean and standard deviation of those topics' call durations in order to more accurately replicate the call center.  The departments specifically included in the model account for 90\% of the total call volume in the original source data, while the ``other" category constitutes the remaining 10\%.  This is illustrated in Figure 3.

	\paragraph{Self-Service}
	
One of the purposes of this project was to evaluate the impact of adding a self-service option that would eliminate the need to speak to an agent for certain call topics.  This was added as an extension of the base model, but was run separately from the base model for the use of direct comparison of the results.  Depending on the assigned call topic, callers have the option to self-service, with multiple routes leading to the agent queue in the event the caller has trouble with self-service and needs an operator.


\begin{figure}[h]
	\includegraphics[scale=.53]{Calls_Department_sim.png}
	\caption{Call distribution by department from source data}
\end{figure}

	\paragraph{Language}

Each call and voicemail is assigned a language of either Spanish or English.  The actual call center deals in both of those language, and it has Spanish-speaking operators available.  In order to reflect this, the model's resources all speak English, with some able to speak Spanish as well.  Therefore, arrivals will be routed into two queues:  one for English, and one for Spanish.  Historical data regarding this distribution was not immediately available, so this is modeled as an estimated parameter that was generated from the population demographics of South Bend.

	\paragraph{Priority}

All incoming calls are assigned a medium priority by default and all voicemails are given a low priority by default (as they can be answered and returned as time permits).  After speaking to an agent, calls needing escalation to a supervisor are given the highest priority (in the event they need to cycle back through the queue again).


\section{Verification \& Validation}

	\subsection{Verification Techniques}

The model and simulation was verified using two primary techniques.  First, the model was tested and run in stages, using simplified or constant parameters.  For example, voicemails were not created and queues were eliminated in order to most simply ensure the call arrival and attribute assignment worked as expected.  A similar process was repeated with voicemails and excluding call arrivals.  Resources (operators and supervisor) were reduced to one at a time and stepped through in order to ensure proper call processing treatment.  

\par

The model was also verified by ensuring the output matched the estimated results produced from a static analysis.  As an example, this includes the comparison of the total number of calls, number of dropped calls, and number of operator breaks for a given simulation period.  With randomness removed from the simulation, it is possible to ensure the numbers produced from these processes are close to what is expected.
	
	
	\subsection{Validation Plan}
	
The model and simulation's results were ultimately validated using historical data.  Since several years' worth of call center data was available, the results of the base simulation were compared to this.  The number of call arrivals was created from the historical data, so this can be compared as well.  The operator schedule was provided by call center management, allowing for actual validation for these values.  Historical data also exists for percent of calls abandoned, which was also be directly compared.
	
\section{Simulation}

	\subsection{Method}
	
A call center is modeled as a discrete event model, leading to the logical choice of Arena for this simulation.  Arena excels in creating and monitoring entities, while providing a visual representation of their flow through a system.  Arena allows for efficient statistic and metric collection, which can be organized and analyzed to answer any questions initially posed for the simulation.

\par

Arena also features built-in modules that vastly simplify actions that would need to intricately programmed in a high-level language.  A prime example of this is Arena's schedule option, which is a built-in option that was leveraged for two aspects of this project:  the call arrival schedule and operator schedule.  By using this included module, the simulation was able to be developed as close to ``real-life" as possible in these areas.
	
	\subsection{Attributes \& Resources}

Three main attributes are assigned to entities upon entry to the system:  topic, language, and priority.  Topic serves as a determinant of the call duration distribution, and is pulled from historical data.  Language is assigned to be either English or Spanish, and the distribution is based on the South Bend Hispanic population percentage.  While other residents surely speak other native languages, the call center only deals internally with English or Spanish, and this simulation is set up to reflect this.  Priority is assigned as a ``medium" for incoming calls, ``low" for voicemails to be returned, and ``high" for calls routed to or from a supervisor.

\par

Two resources exist in this call center model:  call center liaisons (operators) and supervisors.  The operator schedule (and thus total resources at any given simulation time) was based on the historical data as opposed to being an estimated parameter.  This schedule can be seen in Figure 4.  In keeping with typical call center model processes as described by Zhang, these operators are treated as having the same skill level, and thus do not influence the handling of a call \cite{zhang}.  Supervisors handle calls identified as problematic, but are again treated as if they are of the same skill level.  Operator lunches and personal breaks are also based on the schedule provided.

	\begin{figure}[h]
		\includegraphics[scale=.3]{schedule2.jpg}
		\caption{Call center operator schedule}
	\end{figure}
	
	\subsection{Parameters}
	
	
	This simulation utilizes an accurate representation of arrivals as calculated from the aforementioned open-source data.  Mean call arrivals were identified down to the hour for each weekday.  This allows for the simulation to not only pick up on historical call patterns by hour, but also the differences between each day.  A Poisson distribution, broken down by this hourly data, was used to model the arrivals, allowing for the arrivals to be scheduled as such in Arena.  This will base the actual volume of calls in line with prior historical trends, as opposed to applying a rough blanket estimate.
	
	\par
	
	Several other estimated parameters exist in this model and simulation.  These included callers getting a busy signal (and the corresponding retry rate), callers who experience a problem and need to speak with a supervisor, and whether calls who spoke to a supervisor need to be transferred back to an agent to complete a task.  These parameters were estimated through the use of research and experimentation in an effort to represent the center as closely as possible, since no data existed for these.  However, they were set at levels where total calls abandoned and handled were similar to the historical data.
	
	\subsection{Run Settings \& Replications}
	
	 The call center is open five days a week, from 7:30 AM to 5:30 PM.  Therefore, the simulation was run for 100 replications of five ten-hour days.  Each replication serves to represent an ``average" week for the call center.  In this case, 100 replications should be enough in order to obtain a reliable simuation of the call center.
	 
	 
	\subsection{Adjustments for Self-Serve Option Model}
	
	The base model was expanded upon into a second model in order to evaluate the effectiveness of adding a self-service alternative before speaking to an agent.  The ``route" for calls to self-service utilized empirical research regarding consumer attitudes for automated systems.  A survey by Nuance Enterprise suggests that up to 75\% of consumers see self-service as a convenient option when available while 67\% said they preferred self-service over dealing with a company representative \cite{webblog}.  The simulation utilizes the 67\% preference to determine the number of call arrivals that will be routed into self-service options, given the customer has been assigned a topic that is self-serviceable.  A self-serviceable topic is defined as a percentage of callers with a ``Water Works" and ``Solid Waste" topic, since many of these calls can be handled via a touch-tone or voice system.  All remaining calls were routed to the operator queue discussed in the original model.

\section{Experimentation \& Results}

	\subsection{Verification \& Validation}
	
	  An average week of simulation yielded a total of 2,586 calls, with static analysis suggesting an average of 2,505.  The simulation also yielded the correct number of breaks given the number of full-time and part-time employees that worked on a given replication.  The static analysis was computed by taking the sum of arrivals from the arrival schedule (a Poisson distribution) with no regard for randomness.  These values affirm the results being generated from the simulation.  For validation purposes, historical data suggests a total of 2,650 calls per week.  Again, these numbers can be seen as affirmation for the purposes of this simulation.
	  
	  \par
	  
	  Similarly, the number of breaks taken for the simulation match both the static analysis (serving as verification) and the actual operator schedule (serving as validation).  Other key parameters, such as call duration and the number of voicemails created, match what was expected.  Given this information, along with more intricate verification of the model during construction, provides confidence that the simulations performed are valid.
	  
	  
	\subsection{Number of Operators}
	
	An evaluation of the number of operators on schedule was the first question addressed through simulation.  For this experimentation, three primary metrics were evaluated:  number of calls abandoned (hang-ups), average operator utilization, and average waiting time.  The number of calls abandoned metric, it should be noted, is based on a completely estimated parameter which has calls that have spent over 15 minutes in the queue ``hang up."  This is set as a repeating manager in Arena that checks the queue every 5 minutes.  While this is not how hang ups are determined in real-life, it does provide call abandonment numbers that are in line with historical data.  Additionally, since it is consistent among these trials, it still serves as a useful measure of the impact of schedule changes.
	
	\par
	
	The base number of operators, taken from the provided schedule in Figure 4, serves as the control for this exploration.  From this starting point, additional full-time operators were added and subtracted to gauge the effect.  The effects of adding a specific part-time schedule or such was not evaluated, since it would require input from the city for a specific request.  However, these simulations are set up as such that a specific inquiry could be easily explored.
	
	\par
	
	The simulation was run with the base operator schedule, -1, +1, and +2.  Calls abandoned and operator utilization can be seen in Figure 5.  Both of these metrics reacted as expected, with less hang-ups and lower operator utilization as the number of operators increased.  Wait time was an interesting metric, in that it did not vary significantly at all with the change in operators, varying by 0.02 hours.  This can most likely be attributed to the customer patience parameter; as callers hang up, the queue advances.  It seems that over replications, this effect evens out regardless of the number of operators (within reason).
	
	\begin{figure}[h]
		\includegraphics[scale=.5]{num_ops_plot.png}
		\caption{}
	\end{figure}
	
	
	\subsection{Self-Service Call Option}

	The addition of a self-serve option was explored.  In this experimentation, only the percentage of eligible callers who choose the self-service option was adjusted; all other parameters were held constant.  Similarly, the number of operators was held to the current, actual schedule as provided by the call center.  This was done so in order to gauge the amount of consumers who adopt the self-serve option that would show a benefit to the call center.
	
	\par
	
	This experiment was run by utilizing Arena's Process Analyzer tool on the self-serve model.  The percent of customers (who have an eligible topic) was established as the control.  It was observed from 35\% to 85\%, in 5\% increments, as well as the literature-supported 67\% level.  Customer average wait time, number in the main operator queue, and operator utilization were set as the response variables.  Since this analysis is purely hypothetical, any sort of maximum self-service support threshold was not accounted for.
	
	\par
	
	Overall, this experimentation worked as expected.  As the number of callers who chose the self-service option increased, the wait time, queue length, and agent utilization all experienced linear decreases.  A visualization of the queue statistics can be seen in Figure 6.  Furthermore, the decrease in operator utilization can be seen in Figure 7.  It should be noted that in these trials, the graphs do show a definitive decrease in all metrics.
	
	\par
	
	The theorized 67\% utilization rate for the self-service option was then explored with an increase in mean call arrivals.  Through this analysis, it was observed that the self-service model with approximately a 20\% increase in total call arrivals exhibits agent utilization levels in range of the current state (base model) of the call center without the self-service option ($\approx$ 40\%).

\begin{figure}[h]
	\includegraphics[scale=.5]{self_serve_results.png}
	\caption{}
\end{figure}


\begin{figure}[h]
	\includegraphics[scale=.5]{op_self_serve.png}
	\caption{}
\end{figure}



\section{Discussion}

The results of both experiments provided clear metrics on which to base a decision.  First, it appears that the call center could add or remove a full-time operator without a substantial cost or customer service impact.  Removing an agent would change hang-ups per day from 8 to 16 (still a small percentage of the 400-500 daily calls received), and it would lead to a 6\% increase in agent utilization.  Given this, lessening the staffing levels by one full-time agent may well be a smart decision.  Adding one or two more operators reduces calls abandoned, but at a diminishing rate when compared to the loss of utilization.

\par

Secondly, the self-serve simulation experiment shows that operator utilization decreases in a linear fashion depending on the number of callers who can and do choose the self-service option, all else held equal.  Similarly, wait time and queue length decrease as well.  As expected, a self-serve option that even a small percentage of customers use could clearly make the case for removing one operator (as supported above, even without a self-serve option).  However, the true effectiveness of adding such a system would take into other factors, such as start-up and maintenance costs and customer education and marketing.  The extension of the self-service model analysis shows the increased amount of volume the call center can handle without necessitating hiring more staff.  Depending on the city's forecasts for the center, this may be a convincing argument to add this self-service system.

\par

Both of these experiments include estimated parameters, particularly the hypothetical self-service model.  As a result, neither of these experiments definitively answer/attempt to answer any of these questions until they are compared to internal data for accuracy.  Such data would replace estimates with more accurate numbers and distributions; then, a simulation could be run for a definitive question.


\section{Conclusion}

Overall, this analysis resulted in the building of an accurate, historically validated model that was modified and used to provide results to two main questions.  While call centers are a frequently modeled system, this project was unique since it is based on an actual, operating center.  The inclusion of as much historical data as possible leads this analysis from the realm of theoretical into a practical and useful application.  The primary goal of this project was to create a model as similar as possible to the actual call center; this was achieved.  The questions explored, while simplistic in nature, would surely be of interest to city officials when it comes to budgets and annual planning.


	\subsection{Challenges}

	This model development and simulation presented two main challenges that had to be overcome in order to produce a successful project.  First, while many parameters for the model were either contained in or could be deduced from the historical data, several were not.  These included customer patience levels, number of supervisors (and the number of calls they handled), and the number of voicemails necessitating a return call.  Determining appropriate levels for these parameters involved some trial and error, and perhaps may not still be completely accurate.  However, the confidence held in the call arrival rate, duration distributions, and operator schedule is exceptionally strong due to this historical data analysis.  Since these components are arguably the most important aspects of the call center, the model is not particularly sensitive to the estimated parameters.
	
	\par
	
	A second main challenge was determining when the model's complexity was sufficient in order to explore the posed questions and avoiding an overly-complicated or unbalanced model.  With an actual call center as the subject, this model could have easily become so complicated as to be inefficient and more error-prone (harder to test and debug).  A prime example of this is the assignment of call topics to determine the call duration distribution.  Twenty-six departments, each with their own duration statistics, exist in the historical data.  However, the final model incorporates only the six most common departmental topics, with the nineteen others aggregated as a weighted average.  While it is tempting to replicate the historical data as accurately as possible, including all twenty-six options with their statistics would have created an unnecessarily complex section of the model, leading to imbalance.  This would have made verification of the corresponding modules nearly impossible.
	
	\par
	
	These two challenges were successfully overcome in the development of the model and its corresponding simulation.  While an imperfect representation of the actual call center, like any model, it accurately captures the most important aspects in order to develop worthwhile results.  With time, more challenges could be overcome and identified, but the current model functions well for the level of the desired analysis given the information and data available.

	\subsection{Opportunities for Further Study}

	This model lays a successful base for future analysis or study.  If more data, beyond the publicly available data, was to be acquired, the model and its parameters could be defined even more accurately.  Similarly, if further time was dedicated, the model could be designed to represent a complete year for the call center, including holidays, seasonal call and topic flucuations, and possibly fluctuating staffing levels.  However, the work to create such a specific model would require a goal that has not been defined.  The current model will be provided to the city of South Bend.  Since Arena is publicly available, the business analytics department could perhaps retain the model and build on it/use it as the need arises.











\newpage
\clearpage
\pagenumbering{gobble}
\begin{thebibliography}{12}
	


\bibitem{baraka}
Hesham A. Baraka, Hoda Baraka, and Islam Hamdy EL-Gamily. 2013. Assessing call centers’ success: A validation of the DeLone and Mclean model for information system. \textit{Egyptian Informatics Journal} 14, 99-108.

	\bibitem{chen}
Chen, R-R., Chiang, Y., Chong, P.P., Lin, Y-H. \& Chang, H-K. (2011). Rough set analysis on call center metrics. Applied Soft Computing, 11, 3804-3811.

\bibitem{brown}
Lawrence Brown, Noah Gans, Avishai Mandelbaum, Anat Sakov, Haipeng Shen, Sergey Zeltyn, and Linda Zhao. Statistical Analysis of a Telephone Call Center: A Queueing-Science Perspective. \textit{Journal of the American Statistical Association} 100 (469), 36-50.

\bibitem{zhang}
Xiaowei Zhang, L. Jeff Hong, and Jiheng Zhang. 2014. Scaling and Modeling of Call Center Arrivals. In \textit{Proceedings of the 2014 Winter Simulation Conference (WSC ’14)}. IEEE Press, Piscataway, NJ, 467-485. 

	\bibitem{bapat}
Vivek Bapat and Eddie B. Pruitte Jr. 1998. Using Simulation in Call Centers. In \textit{Proceedings of the 30th.  1998 Winter Simulation Conference (WSC ’98)}. IEEE Computer Society Press, Los Alamitos, CA, 1395-1400.

\bibitem{saltzman}
Robert Saltzman and Vijay Mehrotra. 2007. Managing Trade-offs in Call Center Agent Scheduling: Methodology and Case Study. In \textit{Proceedings of the 2007 Summer Computer Simulation Conference (SCSC ’07)}. Society for Computer Simulation International, San Diego, CA, 643-651.


\bibitem{saltzman2}
Robert Saltzman and Vijay Mehrotra. 2001. A Call Center Uses Simulation to Drive Strategic Change. \textit{Interfaces} 31 (3). DOI: https://doi.org/10.1287/inte.31.3.87.9632.

\bibitem{mandelbaum}
Avishai Mandelbaum, Anat Sakov, and Sergey Zeltyn. 2001. Empirical Analysis of a Call Center. Technion Israel Institute of Technology, Israel.

\end{thebibliography}

\end{document}